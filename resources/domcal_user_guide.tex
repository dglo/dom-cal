\documentclass[10pt]{article}
\usepackage{graphics}
\usepackage{indentfirst}
\usepackage{mathrsfs}
\textheight 8.5in \textwidth 6.5in \oddsidemargin 0in \topmargin -.35in

\begin{document}
                                                                                
\baselineskip=16pt
\def\Z{{\bf Z}}
                                                                                
\newcommand{\infinity}{\infty}
                                                                                
\title{DOM-Cal Users' Guide}
\author{John Kelley and Jim Braun}
\maketitle
                                                                                
\newpage

\section{Purpose}
%introduction

\section{DOM-Cal}
%description, guide to using dom-cal

\subsection{Pulser Calibration}

\subsection{ATWD Calibraton}

\subsection{Amplifier Calibration}

\subsection{ATWD Frequency Calibration}

\subsection{DOM-Cal Output}

DOM-Cal writes calibration results to the DOM flash filesystem in a binary
file named "calib\_data".  The format of this file is described in detail in
table \ref{tbl:binary} and table \ref{tbl:linearfit}.

\begin{table}
\begin{tabular}{|p{4cm}|p{4cm}|p{4cm}|}
\hline
Quantity & Datum Size & Comments \\
\hline
Version & Short (16 bits) & Version number -- use to determine endianness ( version < 256 ) \\
\hline
Record Length & Short & Length of domcal binary record \\
\hline
Date -- Day & Short & Day calibration was performed \\
\hline
Date -- Month & Short & Month calibration was performed \\
\hline
Date -- Year & Short & Year calibration was performed \\
\hline
Unused & Short & Not used \\
\hline
DOM ID hi bits & Int (32 bits) & DOM ID in true hex format ( not ASCII ) \\
\hline
DOM ID low bits & Int & DOM ID in true hex format ( not ASCII ) \\
\hline
Temperature & Float (32 bits) & DOM temperature in Kelvin \\
\hline
Initial DAC values & Short[16] & Initial values of DAC 0-15 \\
\hline
Initial ADC values & Short[24] & Initial values of ADC 0-23 \\
\hline
FADC calibration pedestal & Short & \\
\hline
FADC calibration gain & Short & \\
\hline
FE pulser calibration & LinearFit & Calibration of internal pulser \\
\hline
ATWD gain calibration & LinearFit[2][3][128] & Calibration of atwd 0-1, channels 0-2, bins 0-127,
such that member 128 is atwd 0, channel 0, bin 127,
and member 129 is atwd 0, channel 1, bin 0, etc. \\
\hline
FE amplifier calibration & Float[6] & Channel 0 amplification and error,
channel 1 amplification and error, channel 2 amplification and error. \\
\hline
ATWD frequency calibration & LinearFit[3] & Frequency calibration for atwd 0, atwd 1. \\
\hline
\end{tabular}
\caption{DOM-Cal binary output format}
\label{tbl:binary}
\end{table}

\begin{table}
\begin{tabular}{|p{4cm}|p{4cm}|p{4cm}|}
\hline
Quantity & Datum Size & Comments \\
\hline
Slope & Float & Slope parameter of linear fit \\
\hline
Y-intercept & Float & Y-intercept parameter of linear fit \\
\hline
R-squared & Float & Quality of fit \\
\hline
\end{tabular}
\caption{Binary format of linear fit}
\label{tbl:linearfit}
\end{table}

\section{Java Interface to DOM-Cal}

DOM-Cal may be easily accessed through the icecube.daq.domcal java package found
in the dom-cal project.  Implimentations to read binary calibration data from a DOM
or initiate a calibration cycle and read back calibration data are available.  DOM
servers, such as dtsx and domserv, are required to read calibration data with the
java client.  To read calibration data, type
\begin{verbatim}
java -cp {$BFD_HOME}/lib/dom-cal.jar icecube.daq.domcal.DOMCal [host] [port]
[output directory]
\end{verbatim}
to download calibration data from a DOM on \$host at \$port and store xml-formatted
output in \$output directory.  The xml calibration file is named by domId.  In many
cases, it is advantageous to read calibration from many DOMs ant once from a sequence
of ports.  To read calibration data from many ports, type
\begin{verbatim}
java -cp {$BFD_HOME}/lib/dom-cal.jar icecube.daq.domcal.DOMCal [host] [port]
[num ports] [output directory]
\end{verbatim}
where \$num ports is the number of sequential ports to scan above \$port.  The ports
need not be sequential, i.e. if three DOMs are available on ports 5000, 5003, and 5008,
setting \$port = 5000 and \$num ports = 9 would access all three DOMs.  To initiate
calibration, add the string "calibrate dom" at the end of your command line arguments:
\begin{verbatim}
java -cp {$BFD_HOME}/lib/dom-cal.jar icecube.daq.domcal.DOMCal [host] [port]
[num ports] [output directory] calibrate dom
\end{verbatim}
would attempt to calibrate and read out \$num ports DOMs, starting from \$port.

\subsection{XML Calibration Output}

The XML calibration output is much more user-friendly then the DOM-resident binary
file.  Figure \ref{fig:xml} is an example output file with the ATWD calibration,
DAC, and ADC values truncated.
An actual output file contains 2x3x128 = 768 ATWD calibration linear fits.

\begin{figure}
\begin{footnotesize}
\begin{verbatim}
<domcal>
    <date>6-11-2004</date>
    <domid>f771bb4dce28</domid>
    <temperature format="Kelvin">238.9725</temperature>
    <dac channel="0">850</dac>
    .
    .
    <dac channel="15">0</dac>
    <adc channel="0">95</adc>
    .
    .
    <adc channel="23">0</adc>
    <pulser>
        <fit model="linear">
            <param name="slope">9.181886E-5</param>
            <param name="intercept">3.7924567E-4</param>
            <regression-coeff>0.9999254</regression-coeff>
        </fit>
    </pulser>
    <atwd id="0" channel="0" bin="0">
        <fit model="linear">
            <param name="slope">-0.0062284535</param>
            <param name="intercept">5.0744777</param>
            <regression-coeff>0.4187727</regression-coeff>
        </fit>
    </atwd>
    .
    .
    <atwd id="1" channel="2" bin="127">
        <fit model="linear">
            <param name="slope">-0.0022831585</param>
            <param name="intercept">2.6237197</param>
            <regression-coeff>0.9999343</regression-coeff>
        </fit>
    </atwd>
    <fadc parname="pedestal" value="2112"/>
    <fadc parname="gain" value="2184"/>
    <amplifier channel="0">
        <gain error="1.5559148">-97.44885</gain>
    </amplifier>
    <amplifier channel="1">
        <gain error="0.8450469">-52.4845</gain>
    </amplifier>
    <amplifier channel="2">
        <gain error="0.119607404">-8.216592</gain>
    </amplifier>
    <atwdfreq chip="0">
        <fit model="linear">
            <param name="slope">-2.282832E-4</param>
            <param name="intercept">4.4375844</param>
            <regression-coeff>0.43570894</regression-coeff>
        </fit>
    </atwdfreq>
    <atwdfreq chip="1">
        <fit model="linear">
            <param name="slope">0.012642916</param>
            <param name="intercept">3.7248025</param>
            <regression-coeff>0.9978187</regression-coeff>
        </fit>
    </atwdfreq>
</domcal>
\end{verbatim}
\end{footnotesize}
\caption{Sample XML calibration output}
\label{fig:xml}
\end{figure}
\end{document}
