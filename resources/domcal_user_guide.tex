\documentclass[10pt]{article}
\usepackage{graphics}
\usepackage{indentfirst}
\usepackage{mathrsfs}
\textheight 8.5in \textwidth 6.5in \oddsidemargin 0in \topmargin -.35in

\begin{document}
                                                                                
\baselineskip=16pt
\def\Z{{\bf Z}}

\newcommand{\infinity}{\infty}

\title{DOM-Cal Users' Guide}
\author{John Kelley and Jim Braun}
\maketitle

\newpage

\section{Purpose}
%introduction

DOM-Cal is stand-alone application which runs directly on the DOM mainboard
and calibrates the DOM front-end electronics.  The calibration constants
obtained with DOM-Cal allow one to translate the DOM's digital output
into voltage and time waveforms.  Furthermore, DOM-Cal's high-voltage
calibration measures the PMT gain at various voltage settings.  

\section{DOM-Cal}
%description, guide to using dom-cal

The DOM-Cal application resides on the DOM flash filesystem as the file
\$domcal.  The application can be run directly from the iceboot
command-line, using the command 

\begin{verbatim}
> s" domcal" find if exec endif
\end{verbatim}

The application will then ask the user for the date (the DOM has no concept
of the date or time of day), and if a high-voltage gain calibration is to
be performed (in which case the PMT high voltage will be enabled).  

DOM-Cal can also be run through a Java interface; see section \ref{java}.

\subsection{Pulser Calibration}

The DOM pulser produces SPE-like waveforms to test and calibrate mainboard
functionality.  Pulser amplitude is controlled through a DAC setting,
DOM\_HAL\_DAC\_INTERNAL\_PULSER ( DAC 12 ).  The pulser calibration
produces a linear relation between DAC setting and pulse amplitude in Volts.

\subsection{ATWD Calibraton}

The two Analog Transient Waveform Digitizers (ATWDs) function as the
primary signal digitizers for the DOMs.  The output of the ATWD consists of
128 samples of the input signal voltage, with each sample level being an
integer from 0 to 1023.  

The ATWD calibration produces a set of 768 (2x3x128) linear fits, one for
each ATWD (0 and 1), channel (0, 1, and 2), and sample bin (0-127).  These
fits translate the ATWD level to an input voltage, before any
channel-specific amplification.  Because the fit parameters vary for each
sample bin, the unique pedestal pattern of each ATWD is automatically taken
care of.  

The calibration uses the ATWD front-end bias voltage
(DOM\_HAL\_DAC\_PMT\_FE\_PEDESTAL) to vary the baseline voltage of the
ATWD, independent of any channel amplification.  CPU-triggered waveforms
are recorded for a number of different bias voltages, and using a known
relationship between the bias DAC setting and true bias voltage, fits for all
of the ATWD bins are calculated.

\subsection{Amplifier Calibration}

Each of the three signal channels of the ATWDs has a given amplification of
the input voltage: nominally 16x for channel 0, 2x for channel 1, and 0.25x
for channel 2.  The amplifier calibration determines the mean gain factors
for each channel, along with the standard error of each gain.

The calibration uses the pulser to produce reasonably-sized waveforms for
each channel.  The maxima of a number of discriminator-triggered waveforms
are recorded, and using the pulser calibration and the ATWD calibration
previously obtained, translated to a voltage.  The mean gain and the error
on this mean are then calculated.

\subsection{ATWD Frequency Calibration}

ATWD frequency is controlled by DAC settings DOM\_HAL\_DAC\_ATWD0\_TRIGGER\_BIAS and
DOM\_HAL\_DAC\_ATWD1\_TRIGGER\_BIAS ( DACs 0 and 4, respectively ).  ATWD frequency
calibration produces a linear relation between DAC setting and ratio between ATWD
frequency and the mainboard ~20MHz oscillator.

\subsection{High Voltage Gain Calibration}

As an option, DOM-Cal can calibrate the gain of the PMT as a function of
the high voltage.  DOM-Cal will return a linear fit of $\log(gain)$ versus
$\log(voltage)$.  Additionally, DOM-Cal returns the peak-to-valley ratio at
each of the voltage settings at which a good charge histogram was achieved.  

To find the gain at a particular high voltage, a number of
discriminator-triggered waveforms are taken.  Using the results of the
previous calibrations, the waveform is integrated, and the charge in pC is
recorded in a histogram.

The histogram is fit via nonlinear (Levenberg-Marquardt) regression to the
functional form $A\ e^{-Bx}\ +\ C\ e^{-E(x-D)^2}$.  The single
photoelectron (SPE) peak (and thus the gain) is reasonably approximated by
the parameter $D$, while the valley between the noise peak and the SPE peak
is found via iterative (Newton-Raphson) search.  A few heuristics are used
to determine if the fit is sane, and only then is the extracted data used
for the rest of the calibration.

After the gain is obtained for each high voltage setting, a linear
regression is performed on $\log(HV)$ vs $\log(gain)$.

\subsection{DOM-Cal Output}

DOM-Cal writes calibration results to the DOM flash filesystem in a binary
file named "calib\_data".  The format of this file is described in detail in
table \ref{tbl:binary} and table \ref{tbl:linearfit}.

\begin{table}
\begin{tabular}{|p{4cm}|p{4cm}|p{4cm}|}
\hline
Quantity & Datum Size & Comments \\
\hline
Version & Short (16 bits) & Version number -- use to determine endianness ( version < 256 ) \\
\hline
Record Length & Short & Length of domcal binary record \\
\hline
Date -- Day & Short & Day calibration was performed \\
\hline
Date -- Month & Short & Month calibration was performed \\
\hline
Date -- Year & Short & Year calibration was performed \\
\hline
Unused & Short & Not used \\
\hline
DOM ID hi bits & Int (32 bits) & DOM ID in true hex format ( not ASCII ) \\
\hline
DOM ID low bits & Int & DOM ID in true hex format ( not ASCII ) \\
\hline
Temperature & Float (32 bits) & DOM temperature in Kelvin \\
\hline
Initial DAC values & Short[16] & Initial values of DAC 0-15 \\
\hline
Initial ADC values & Short[24] & Initial values of ADC 0-23 \\
\hline
FADC calibration pedestal & Short & \\
\hline
FADC calibration gain & Short & \\
\hline
FE pulser calibration & LinearFit & Calibration of internal pulser \\
\hline
ATWD gain calibration & LinearFit[2][3][128] & Calibration of 
ATWD 0-1, channels 0-2, bins 0-127,
such that member 128 is atwd 0, channel 0, bin 127,
and member 129 is atwd 0, channel 1, bin 0, etc. \\
\hline
FE amplifier calibration & Float[6] & Channel 0 amplification and error,
channel 1 amplification and error, channel 2 amplification and error. \\
\hline
ATWD frequency calibration & LinearFit[3] & Frequency calibration for atwd 0, atwd 1. \\
\hline
\end{tabular}
\caption{DOM-Cal binary output format}
\label{tbl:binary}
\end{table}

\begin{table}
\begin{tabular}{|p{4cm}|p{4cm}|p{4cm}|}
\hline
Quantity & Datum Size & Comments \\
\hline
Slope & Float & Slope parameter of linear fit \\
\hline
Y-intercept & Float & Y-intercept parameter of linear fit \\
\hline
R-squared & Float & Quality of fit \\
\hline
\end{tabular}
\caption{Binary format of linear fit}
\label{tbl:linearfit}
\end{table}

\section{Java Interface to DOM-Cal}
\label{sec:java}
DOM-Cal may be easily accessed through the icecube.daq.domcal java package found
in the dom-cal project.  Implimentations to read binary calibration data from a DOM
or initiate a calibration cycle and read back calibration data are available.  DOM
servers, such as dtsx and domserv, are required to read calibration data with the
java client.  To read calibration data, type
\begin{verbatim}
java -cp {$BFD_HOME}/lib/dom-cal.jar icecube.daq.domcal.DOMCal [host] [port]
[output directory]
\end{verbatim}
to download calibration data from a DOM on \$host at \$port and store xml-formatted
output in \$output directory.  The xml calibration file is named by domId.  In many
cases, it is advantageous to read calibration from many DOMs ant once from a sequence
of ports.  To read calibration data from many ports, type
\begin{verbatim}
java -cp {$BFD_HOME}/lib/dom-cal.jar icecube.daq.domcal.DOMCal [host] [port]
[num ports] [output directory]
\end{verbatim}
where \$num ports is the number of sequential ports to scan above \$port.  The ports
need not be sequential, i.e. if three DOMs are available on ports 5000, 5003, and 5008,
setting \$port = 5000 and \$num ports = 9 would access all three DOMs.  To initiate
calibration, add the string "calibrate dom" at the end of your command line arguments:
\begin{verbatim}
java -cp {$BFD_HOME}/lib/dom-cal.jar icecube.daq.domcal.DOMCal [host] [port]
[num ports] [output directory] calibrate dom
\end{verbatim}
would attempt to calibrate and read out \$num ports DOMs, starting from \$port.

\subsection{XML Calibration Output}

The XML calibration output is much more user-friendly then the DOM-resident binary
file.  Figure \ref{fig:xml} is an example output file with the ATWD calibration,
DAC, and ADC values truncated.
An actual output file contains 2x3x128 = 768 ATWD calibration linear fits.

\begin{figure}
\begin{footnotesize}
\begin{verbatim}
<domcal>
    <date>6-11-2004</date>
    <domid>f771bb4dce28</domid>
    <temperature format="Kelvin">238.9725</temperature>
    <dac channel="0">850</dac>
    .
    .
    <dac channel="15">0</dac>
    <adc channel="0">95</adc>
    .
    .
    <adc channel="23">0</adc>
    <pulser>
        <fit model="linear">
            <param name="slope">9.181886E-5</param>
            <param name="intercept">3.7924567E-4</param>
            <regression-coeff>0.9999254</regression-coeff>
        </fit>
    </pulser>
    <atwd id="0" channel="0" bin="0">
        <fit model="linear">
            <param name="slope">-0.0021576127</param>
            <param name="intercept">2.6414666</param>
            <regression-coeff>0.9999896</regression-coeff>
        </fit>
    </atwd>
    .
    .
    <atwd id="1" channel="2" bin="127">
        <fit model="linear">
            <param name="slope">-0.0022831585</param>
            <param name="intercept">2.6237197</param>
            <regression-coeff>0.9999343</regression-coeff>
        </fit>
    </atwd>
    <fadc parname="pedestal" value="2112"/>
    <fadc parname="gain" value="2184"/>
    <amplifier channel="0">
        <gain error="0.008047878">-19.214722</gain>
    </amplifier>
    <amplifier channel="1">
        <gain error="0.0013110796">-2.8445616</gain>
    </amplifier>
    <amplifier channel="2">
        <gain error="2.8432324E-4">-0.1962693</gain>
    </amplifier>
    <atwdfreq chip="0">
        <fit model="linear">
            <param name="slope">0.012672497</param>
            <param name="intercept">3.5881329</param>
            <regression-coeff>0.9979777</regression-coeff>
        </fit>
    </atwdfreq>
    <atwdfreq chip="1">
        <fit model="linear">
            <param name="slope">0.012642916</param>
            <param name="intercept">3.7248025</param>
            <regression-coeff>0.9978187</regression-coeff>
        </fit>
    </atwdfreq>
</domcal>
\end{verbatim}
\end{footnotesize}
\caption{Sample XML calibration output}
\label{fig:xml}
\end{figure}
\end{document}
