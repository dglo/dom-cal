\documentclass[10pt]{article}
\usepackage{graphics}
\usepackage{indentfirst}
\usepackage{mathrsfs}
\textheight 8.5in \textwidth 6.5in \oddsidemargin 0in \topmargin -.35in

\begin{document}

%% Aliases 
\newcommand{\be}{\begin{equation}} 
\newcommand{\ee}{\end{equation}}
\newcommand{\bea}{\begin{eqnarray}} 
\newcommand{\eea}{\end{eqnarray}}
\newcommand{\bi}{\begin{itemize}}
\newcommand{\ei}{\end{itemize}}

\baselineskip=16pt

\title{DOM-Cal XML Users' Guide\\ Version 7.4}
\author{John Kelley\footnote{jkelley@icecube.wisc.edu}\ \ and Jim Braun\footnote{jbraun@icecube.wisc.edu}}
\maketitle

\newpage
%--------------------------------------------------------------------------

\section{Introduction}

This document describes the result XML file produced by the DOM-resident
calibration software DOM-Cal.  The intent is to allow a user to develop a
reader which will calibrate DOM waveforms, allowing transformation of both raw
and pedestal-subtracted waveforms into physical units such as voltage,
time, and charge.   

This document does not describe the means of calibration.  More information
about the calibration methods is available in the DOM-Cal Users'
Guide, available as part of the dom-cal project.

%--------------------------------------------------------------------------

\section{XML File Description}

The XML elements are presented below in the order in which they appear in
the output file.  

\bi
\item{$<$domcal version=\textit{version}$>$}\\
The root element of the XML file contains the version of DOM-Cal which
produced it as an attribute.  The version is a string of the format
\textit{N.N.N}, with each field denoting major, minor, and patch versions
respectively.  This version matches the release tag of the software.
\bi
\item{$<$date$>$}\\
 The date of the calibration, in DD-MM-YYYY format.

\item{$<$time$>$}\\
 The time of the calibration, in HH:MM:SS format.  Note that the date/time
 are in GMT.

\item{$<$domid$>$}\\
 The DOM mainboard ID (hexadecimal).

\item{$<$temperature format=''Kelvin''$>$}\\
 The DOM temperature at calibration time, measured in K by the mainboard
 temperature sensor.  This is 10-12K above the ambient temperature, unless
 the DOM has just been powered on.  Note that in general, the calibration
 is only valid around this temperature.

\item{$<$dac channel=\textit{channel}$>$}\\
 All mainboard DAC settings (0-15) are recorded.  {\bf IMPORTANT:} in
 general, the calibration should only be considered valid for these DAC
 settings!  In particular, the ATWD operational settings in DACs 1-3 and
 5-7 must  match any data to which this calibration is applied.  The
 sampling speed in DACs 0 and 4 can be adjusted, however.

\item{$<$adc channel=\textit{channel}$>$}\\
 All mainboard ADC values (0-23) are recorded.

\item{$<$frontEndImpedance format=''Ohms''$>$}\\
 The front-end impedance used for the calibration.  The same value must be
 used in any charge calculations which depend on DOM-Cal results.  See
 section \ref{calusage} for more information.

%-----------

\item{$<$discriminator id=''spe''$>$}\\
 The front-end SPE discriminator calibration.  This calibration translates the
 SPE discriminator DAC setting (DAC 9) to the amplified charge (after the
 PMT) corresponding to that trigger level. 
\bi
\item{$<$fit model="linear"$>$}
\bi
\item{$<$param name="slope"$>$}
\item{$<$param name="intercept"$>$}
\item{$<$regression-coeff$>$}
\ei
A linear fit element describing the calibration results.  There are a large
number of these in the DOM-Cal output, all of this form.  The fit element
contains three sub-elements: the slope and intercept of the fit, along with
the $R^2$ coefficient, all in floating-point format. 

This fit provides the following relation:
\be
\mathrm{charge\ trigger\ level\ (pC)} = m \cdot \mathrm{DAC[9]} + b\ .
\ee
\noindent Because this trigger level is relative to a given baseline
determined by the FE (front end) bias voltage DAC (DAC 7), this calibration
is only valid for a given setting of the bias voltage (recorded in the
appropriate $<$dac$>$ element). 

See the $<$pmtDiscCal$>$ result for a refinement of this calibration. 
\ei % End fit

%-----------

\item{$<$discriminator id=''mpe''$>$}\\
 The front-end MPE discriminator calibration.  This calibration translates the
 MPE discriminator DAC setting (DAC 8) to the amplified charge (after the
 PMT) corresponding to that trigger level. 

 Like the SPE discriminator calibration, a linear fit element describes the
 results, providing the following relation:
 \be
 \mathrm{charge\ trigger\ level\ (pC)} = m \cdot \mathrm{DAC[8]} + b\ .
 \ee
 \noindent Because this trigger level is relative to a given baseline
 determined by the FE (front end) bias voltage DAC (DAC 7), this calibration
 is only valid for a given setting of the bias voltage (recorded in the
 appropriate $<$dac$>$ element). 

%-----------

\item{$<$atwd id=\textit{id} channel=\textit{channel} bin=\textit{bin}$>$}\\ 
A series of 768 linear fits which calibrate the ATWDs.  The attributes
indicate the id (ATWD 0 or 1), the channel (0, 1, or 2), and the waveform
bin (0-127) for which the fit it valid.  The linear fit provides
transformation from ATWD counts to voltage (Volts, after channel
amplification): 

\be
\mathrm{V(id, channel, bin)} = m \cdot \mathrm{counts(id, channel, bin)} + b\ .
\ee

For example, to reconstruct a single waveform in ATWD 1, channel 0, the
first step is to use the 128 fits with $id=1$ and $channel=0$ to reconstruct
the voltage for each bin. {\bf IMPORTANT:} the fits for each bin
automatically calibrate out the pedestal pattern of the ATWD.  The
calibration should not be applied as described to pedestal-subtracted
data.  See section \ref{calusage} for more information.  Also note that the
bin number corresponds to the sample in the raw time-reversed ATWD
waveform, so bin 127 is the earliest sample in time, and bin 0 is the latest.

Also note that there is currently no calibration available (or planned) for
the y-axis of the analog multiplexer channels.  The time axis can be
treated just like any other ATWD channel.

%-----------

\item{$<$fadc\_baseline$>$}\\
The FADC baseline calibration records the linear response of the FADC
baseline (in ADC ticks) to the FADC reference DAC setting, DAC 10.  There
is also a smaller dependence of the FADC baseline on the front-end
bias voltage, DAC 7, so this relationship should be considered valid for
the DAC 7 value recorded above in the appropriate $<$dac$>$ element.

\be
\mathrm{FADC\ baseline\ (ticks)} = m \cdot \mathrm{DAC[10]} + b\ .
\ee

%-----------

\item{$<$fadc\_gain$>$}\\
The FADC gain calibration element.
\bi
\item{$<$gain error=\textit{error}$>$}\\
The gain element contains the FADC gain, including the intrinsic gain of
the FADC as well as the channel amplifiers, as a floating-point number.
The error reported as an attribute is the $1\sigma$ error on this mean gain
measurement.  Note that unlike the gain of the ATWD channels, this
calibration element has units (V/tick).
\ei

%-----------

\item{$<$fadc\_delta\_t$>$}\\
The FADC time offset calibration.
\bi
\item{$<$delta\_t error=\textit{error}$>$}\\
The delta\_t element contains the FADC time offset, in ns, from ATWD
waveform start.  The number reported is negative, as the FADC waveform starts
earlier in time than the ATWD waveform.   The error reported as an attribute
is the $1\sigma$ error on this mean delta\_t measurement.  This calibration is
valid only for the DOMApp FPGA (used in pDAQ), not the STF FPGA (used
in testDAQ).  
\ei

%-----------

\item{$<$atwd\_delta\_t$ id=\textit{id}>$}\\
The ATWD time offset calibration.
\bi
\item{$<$delta\_t error=\textit{error}$>$}\\
The delta\_t element contains the ATWD time offset, in ns, resulting from slightly different 
delays in ATWD digitization start.  The ATWD used in the transit time calibration will have
an offset of zero, while the other delta\_t element provides a small (a few nanoseconds) correction
for the other ATWD.  See section \ref{calusage} 
for more information.  The error reported as an attribute
is the $1\sigma$ error on this mean delta\_t measurement.
\ei

%-----------

\item{$<$amplifier channel=\textit{channel}$>$}\\
The amplifier gain calibrations for the three ATWD signal channels (0, 1, and 2).
The calibrations are valid both for ATWD 0 and ATWD 1.  The reconstructed
waveforms from the ATWD bin fits need to be divided by the channel gain to
get the PMT output voltage.
\bi
\item{$<$gain error=\textit{error}$>$}\\
The gain element contains the channel gain as a floating-point number.  The
gain is negative to reconstruct the waveform as positive-going.  The error
reported as an attribute is the $1\sigma$ error on this mean gain measurement.  
\ei

%-----------

\item{$<$atwdfreq atwd=\textit{atwd}$>$}\\
The ATWD time calibrations for ATWD 0 and 1.  This can be used to translate
the ATWD waveform x-axis into time.  Unlike other ATWD calibrations,
the relation between DAC setting and sampling speed is not quite linear,
but is nicely described quadratically with the following fit element:

\bi
\item{$<$fit model="quadratic"$>$}
\bi
\item{$<$param name="c0"$>$}
\item{$<$param name="c1"$>$}
\item{$<$param name="c2"$>$}
\item{$<$regression-coeff$>$}
\ei
\ei

The $c_n$ fit parameters correspond to the best-fit coefficients of the
$n$th order term.  Thus, the fits relate the sampling speed DAC setting
(DAC 0 for ATWD 0, DAC 4 for ATWD 1) to a sampling speed frequency as
follows: 

\be
\mathrm{sampling\ speed\ (MHz)} = c_0 + c_1 \cdot \mathrm{DAC[0,4]} + c_2
\cdot \mathrm{DAC[0,4]}^2\ .
\ee

This can obviously be inverted to get the time per sample for the
waveform.  Remember that the last sample in the raw waveform is the first
sample in time.

%------------

\item{$<$baseline voltage="0"$>$}\\ 
The first of a number of baseline measurements at various high voltage
settings, which record the remnant baseline ($<$1 mV) after waveforms are
transformed to voltage and the bias level subtracted.  Ideally, this would
be zero, but the baseline has been found to be very sensitive to the
internal state of the DOM.  DOM-Cal precisely determines the baseline in
order to obtain accurate amplifier gain and charge measurements.  Whether
the user decides to use these baselines or determines them dynamically on a
waveform-to-waveform basis is an open question.  Note that these baselines
are amplified voltages for the given channel, so they should be subtracted
before dividing out the channel gain.\\

\bi
\item{$<$base atwd=\textit{atwd} channel=\textit{channel}
  value=\textit{value}$>$}\\ 
The \textit{value} attribute records the baseline offset, in Volts, of
a waveform in the given ATWD and channel.  From the parent element one can
determine that these baselines were measured with the high voltage off.
\ei

These baseline measurements have been obtained using the testDAQ (``stf'')
FPGA.  For baseline measurements which are more applicable to data taken
with the DAQ (``domapp'') FPGA, see the $<$daq\_baseline$>$ calibration element.

%------------

\item{$<$daq\_baseline$>$}\\ 
A series of residual baseline measurements for each ATWD, channel, and bin
obtained using the DAQ (``domapp'') FPGA.  The baseline values (in
Volts) are given as follows:\\

\bi
\item{$<$waveform atwd="\textit{atwd}" channel="\textit{channel}"
  bin="\textit{bin}"$>$\textit{value}$<$/waveform$>$}\\
\ei
These baselines are amplified channel voltages, so they should be
subtracted off before dividing out the channel gain.  These baseline
waveforms are obtained with the DOM in its nominal data-taking state and
should more closely record residual baseline shifts in DAQ data than
the $<$baseline$>$ calibrations.  

%------------

\item{$<$pmtTransitTime num\_pts=\textit{points}$>$}\\
Contains a linear fit element giving the total transit time of the DOM (PMT
transit time plus the delay board), as a function of high voltage.  In
particular, the fit provides the following relation:

\be
\mathrm{transit\ time\ (ns)} = m / \sqrt{V} + b\ .
\ee

Note the $1/\sqrt{V}$ dependence, where $V$ is the high voltage, in Volts.
This calibration can be used to adjust the waveform leading-edge times back
to the time of light reaching the DOM.

The attribute num\_pts indicates the number of high voltages used in the
linear fit (a minimum of two).

%------------

\item{$<$hvGainCal$>$}\\ 
A calibration of the gain of the PMT versus high voltage.  A linear fit
element provides the following log-log relation:

\be
\log_{10} gain = m \cdot \log_{10} V + b\ .
\ee

%------------

\item{$<$pmtDiscCal num\_pts=\textit{points}$>$}\\
Contains a linear fit with the result of the SPE discriminator calibration.
This calibration translates the SPE discriminator DAC setting (DAC 9) to
the amplified charge (after the PMT) corresponding to that trigger level:

\be
\mathrm{charge\ trigger\ level\ (pC)} = m \cdot \mathrm{DAC[9]} + b\ .
\ee

The attribute num\_pts indicates the number of discriminator settings used in
the linear fit.  If present,
this calibration should be used in preference to the $<$discriminator$>$
calibration, as it is more accurate for SPEs and sub-SPE pulses.  

Because the discriminator level is relative to a given baseline
determined by the FE (front end) bias voltage DAC (DAC 7), this calibration
is only valid for a given setting of the bias voltage (recorded in the
appropriate $<$dac$>$ element). 

%------------

\item{$<$baseline voltage=\textit{voltage}$>$}\\
A sequence of baseline measurements as previously described.  These
baseline measurements are at non-zero high voltage settings, one for each
of the voltages used for the HV gain calibration.

Following the baseline measurements are charge histograms taken at various
high voltages:

\item{$<$histo voltage=\textit{voltage} convergent=\textit{true/false}
  pv=\textit{PV ratio} noiseRate=\textit{rate}
  isFilled=\textit{true/false}$>$}\\ 

The charge histogram recorded during the HV gain calibration at a given
voltage.  The attributes of this element are: the high voltage; whether the
gain calibration was successful / convergent and used for the linear fit;
the peak-to-valley ratio for convergent charge histograms; the noise rate
at which the histogram was collected, in Hz; and whether the histogram data
itself follows.

The fit parameters to the histogram are always reported, even if the fit
was not convergent (in which case the last iteration is recorded):

\bi
\item{$<$param name="exponential amplitude"$>$}
\item{$<$param name="exponential width"$>$}
\item{$<$param name="gaussian amplitude"$>$}
\item{$<$param name="gaussian mean"$>$}
\item{$<$param name="gaussian width"$>$}
\ei

As implied above, the charge histogram is fit to a functional form of
$A\ e^{-Bq}\ +\ C\ e^{-E(q-D)^2}$, where $q$ is the amplified charge in
pC.

If the \textit{isFilled} attribute is true, the raw histogram data follows:

\bi
\item{$<$histogram bins=\textit{number of bins}$>$}
\bi
\item{$<$bin num=\textit{bin} charge=\textit{charge}
  count=\textit{count}$>$}\\
The bin data \textit{count} for a given bin \textit{bin} is given along
with the left edge charge value of that bin, in pC.
\ei
\ei % End histo sub-elements

The charge histograms are the final elements in the XML file.  Because
iterative HV/gain calibrations are possible, multiple charge histograms may
be present for a given voltage.  

%------------

\ei % End main element list
\ei % End root list

%--------------------------------------------------------------------------

\section{Calibration Usage}
\label{calusage}

This section gives a very basic description of how to use the calibration
results to transform raw ATWD and FADC waveforms into voltage and time.
More elaborate techniques in waveform reconstruction or feature extraction can
follow but are not covered here.

\subsection{ATWD Waveform Calibration}

\subsubsection{Raw Waveforms}

To transform raw ATWD waveforms (testDAQ data, or DAQ data that has not
been pedestal-subtracted) into voltage: 

\begin{enumerate}

\item{Use the linear fit for each waveform bin (slope $m$ and intercept $b$
  for the appropriate ATWD, channel, and bin) from the $<$atwd$>$
  calibration to convert the ATWD counts into amplified Volts.}

\item{Subtract the bias voltage.  From the bias DAC value (DAC 7),
  determine the front-end bias voltage from this relation:}
\be
\mathrm{V_{bias}(Volts)} = \mathrm{DAC[7]} \cdot 5.0 / 4096.0\ .
\ee

\item{Subtract any remnant baseline.  For testDAQ data, an approximate
  value for the operating voltage can be retrieved from the $<$baseline$>$
  calibration, or the baseline can be determined from the waveform itself.
  For DAQ data, the $<$daq\_baseline$>$ waveform should be subtracted out
  at this point.  Note that the latter is a function of the ATWD bin as well.  
  So, to this point we have:}

\be
\mathrm{V(id, ch, bin)} = m\mathrm{(id, ch, bin)} \cdot \mathrm{wf(id, ch, bin)} + b\mathrm{(id,
  ch, bin)} - \mathrm{V_{bias}} - \mathrm{baseline(id, ch, bin)}\ .
\ee

\item{Divide by the channel gain for the appropriate channel from the
  $<$amplifier$>$ calibration.} 

\end{enumerate}

\subsubsection{Pedestal-Subtracted Waveforms}

The procedure above is applicable to non-pedestal-subtracted
waveforms.  In DAQ data that has been compressed, the ATWD pedestal may have
already been subtracted by the DOM, but a constant offset remains to allow
the waveform to remain positive.  This offset is currently the average of
the pedestal waveform for that ATWD channel.  

To transform pedestal-subtracted waveforms (with offset added) into voltage: 

\begin{enumerate}

\item{Reconstruct the baseline (the pedestal average) for this ATWD and
  channel:}

\be
\overline{p}\mathrm{(id, ch)} = \lfloor\ \frac{1}{N_{bin}}\ \sum_{bin} \left( \frac{\mathrm{V_{bias}} +
\mathrm{daq\_baseline(id, ch, bin)} - b\mathrm{(id, ch, bin)}}
{m\mathrm{(id, ch, bin)}} \right) \rfloor
\ee
\noindent where $m$ and $b$ are the linear fit parameters from the
$<$atwd$>$ calibrations, and $\lfloor\ \rfloor$ is the floor operation
(\textit{i.e.} rounding in truncate mode).  

\item{Subtract this baseline from the waveform, and then use the slope from
  the linear fit from the $<$atwd$>$ calibration to convert
  the ATWD counts into amplified Volts:}
\be
\mathrm{V(id, ch, bin)} = \mathrm{m(id, ch, bin)} \cdot 
(\mathrm{wf(id, ch, bin)} - \overline{p}\mathrm{(id, ch)})\ .
\ee

\item{Subtract any remnant baseline, if necessary.}

\item{Divide by the channel gain for the appropriate channel.} 

\end{enumerate}

\subsubsection{Integrated Charge}

To find the integrated charge of an ATWD waveform one must also:

\begin{enumerate}
\item{Determine the sampling frequency of the ATWD used with the
  $<$atwdfreq$>$ calibration in the result file.}
\item{Convert the summed voltages $V_i$ into amplified charge.  The
  front-end impedance $Z$ (in Ohms) to use is found in the $<$frontEndImpedance$>$
  element.}
\be
\mathrm{charge\ (pC)} = 10^{12} \cdot \frac{1}{Z\ (\Omega)} \cdot
\frac{1}{\mathrm{freq\ (Hz)}} \cdot \sum_i{V_i}\ .
\ee
\item{Divide by the gain of the PMT for the operating high voltage using
  the linear log-log relationship in the $<$hvGainCal$>$ calibration.}
\end{enumerate}

\subsubsection{\label{timecal}Time Calibration}

To translate a leading edge time (or some other feature time) within an ATWD
waveform to a photon hit time at the PMT:

\begin{enumerate}
\item{Determine the bin position (interpolate if you like) of the feature.}
\item{Convert the bin offset within the waveform (recall sample 127 is the
  first sample) into a time $T_{\mathrm{offset}}$ using the $<$atwdfreq$>$
  sampling speed calibration for the appropriate ATWD.}
\item{Add the feature offset time to the ATWD trigger time
  $T_{\mathrm{launch}}$ (GPS time in nanoseconds).}
\item{Subtract the transit time $T_{\mathrm{transit}}$ for the given operating voltage, determined from
  the $<$pmtTransitTime$>$ calibrated relationship.}
\item{Subtract the ATWD offset $\Delta_{\mathrm{ATWD}}$ correction, from the
  the $<$atwd\_delta\_t$>$ calibration.}  The ATWD offset is positive if
  the launch time for this ATWD is later than the ATWD used in the transit
  time calibration -- it can be viewed as a correction to the transit
  time.  
\end{enumerate}

\noindent To summarize, the PMT hit time $T_{\mathrm{hit}}$ can be computed
as follows:

\bea
T_{\mathrm{hit}}\ \mathrm{(ns)} & = & T_{\mathrm{launch}} + T_{\mathrm{offset}} -
(T_{\mathrm{transit}} + \Delta_{\mathrm{ATWD}}) \nonumber\\
& = & T_{\mathrm{launch}} +
\frac{\mathrm{bin}\cdot10^3}{\mathrm{ATWD\ freq.\ (MHz)}} - \left(
\frac{m_{\mathrm{TT}}}{\sqrt{V}} + b_{\mathrm{TT}} +
\Delta_{\mathrm{ATWD}}\right)\ .
\eea

\subsection{FADC Waveform Calibration}

\noindent To calibrate an FADC waveform into voltage:

\begin{enumerate}
\item{Subtract the baseline, using the value obtained from the
  $<$fadc\_baseline$>$ calibration at the operating DAC 10 level.}
\item{Convert the baseline-subtracted waveform into volts, using the (V/tick)
  constant $G$ in the $<$fadc$>$ calibration:}
  \be
  \mathrm{V_i\ (Volts)} = G \cdot (\mathrm{FADC}_i -
  \mathrm{baseline})\ . 
  \ee
\end{enumerate}

\noindent To find the PMT hit time corresponding to a feature within an FADC waveform:

\begin{enumerate}
\item{Determine the FADC sample position of the feature, interpolating if
  you like.}
\item{Determine the offset $T_{\mathrm{offset}}$ in ns from the FADC launch, using the FADC
  sampling frequency of 40.0 MHz.}
\item{Add this offset to the ATWD waveform start time $T_{ATWD}$ (the
  launch time, corrected for waveform transit time and ATWD $\Delta_{\mathrm{ATWD}}$ offset -- see previous
  subsection), and add the FADC inherent time offset $\Delta_{\mathrm{FADC}}$ from
  the $<$fadc\_delta\_t$>$ calibration:}
\bea
\mathrm{FADC\ time\ (ns)} &=& T_{\mathrm{offset}} + 
T_{\mathrm{ATWD}} + \Delta_{\mathrm{FADC}}\ \nonumber\\
&=& \frac{\mathrm{bin}\cdot10^3}{\mathrm{40.0\ MHz}} + T_{\mathrm{launch}}
- (T_{\mathrm{transit}} + \Delta_{\mathrm{ATWD}}) +
\Delta_{\mathrm{FADC}}\ .
\eea
In the calibration results, one will note that $\Delta_{\mathrm{FADC}}$ is
negative, expressing the fact that the FADC 
digitization window starts before that of the ATWD.  
\end{enumerate}

\subsection{Discriminator Setting}

To find the discriminator DAC setting corresponding to some number $N_{PE}$
of photoelectrons (PEs):

\begin{enumerate}
\item{Determine the operating gain of the PMT at the given HV using the
  $<$hvGainCal$>$ log-log fit:}
\be
g = 10^{m\cdot\log_{10}(\mathrm{HV})\ +\ b}\ .
\ee

\item{Determine the SPE discriminator DAC setting for the desired charge by
  inverting the $<$pmtDiscCal$>$ linear fit:}
\be
\mathrm{DAC[9]} = \frac{1}{m} (g \cdot N_{PE} \cdot e \cdot 10^{12} - b)
\ee
where $e$ is the charge of the electron, in Coulombs.
\end{enumerate}

Please feel free to contact the authors if you have any questions about the
use of these calibration results.  

%--------------------------------------------------------------------------

\section{Changes}

This section records changes in the XML file format since the first draft
of this document (starting with DOM-Cal version 5.11).

\bi

\item{7.4}\\
  Added $<$pmtDiscCal$>$ calibration, which refines the SPE discriminator
  calibration by using actual PMT pulses.  If present, this calibration
  should be used in preference to the SPE $<$discriminator$>$ calibration.  \\

\item{7.3}\\
  Fixed sign of $<$atwd\_delta\_t$>$ usage.  Added more detail to hit time
  calculations for both ATWD and FADC.\\

\item{7.2}\\
  Added $<$atwd\_delta\_t$>$ element to record offset between ATWD digitization
  start times.  Switched $<$fadc\_delta\_t$>$ calibration to use the
  ``domapp'' FPGA, increasing the offset by one clock cycle (25 ns).\\
  New: $<$atwd\_delta\_t$>$ calibration records the ATWD digitization start offset
  relative to the transit time calibration.  Both ATWD and FADC delta\_t calibrations use
  the "domapp" FPGA.\\
  Old: No ATWD offset was available.  The FADC time offset calibration used the "stf" FPGA, which 
  has a delay once clock cycle smaller than the "domapp" FPGA.\\ 

\item{7.0}\\
  Added $<$daq\_baseline$>$ element to record residual baseline waveforms
  from the ``domapp'' FPGA.\\
  New: $<$daq\_baseline$>$ waveforms record the residual baseline for each
  ATWD, channel, and bin.\\
  Old: Only testDAQ ``stf'' baselines were available via the $<$baseline$>$
  elements.\\ 

  Client/server architecture changed substantially, in that the DOM doesn't
  write calibration results to the flash filesystem.  Instead, the XML is
  generated on the DOM side and then sent to the surface Java client.  

\item{6.3}\\
  Added front-end impedance element.\\
  New: $<$frontEndImpedance$>$ records the impedance used in the
  calibration.\\
  Old: No impedance information recorded.\\

  Multiple charge histograms per voltage are now possible, if the option 
  to iterate the HV/gain calibration was chosen at runtime.  Also, all
  baseline measurements come before charge histograms.\\
  New: $<$baseline$>$ and $<$histo$>$ elements separated; may be more than
  one $<$histo$>$ per voltage.\\
  Old: $<$baseline$>$ and $<$histo$>$ elements always came in pairs.\\

\item{6.2}\\
  Added time of calibration (GMT).\\
  New: $<$time$>$ HH:MM:SS $<$/time$>$\\
  Old: No time information recorded.\\

  No change to date format \textit{specification}, but note that code
  before this version actually used the American MM-DD-YYYY format, instead
  of DD-MM-YYYY.\\
  New: $<$date$>$ DD-MM-YYYY $<$/date$>$\\
  Old: $<$date$>$ MM-DD-YYYY $<$/date$>$\\

  Added patch version number.\\

\item{6.1}\\ 
  Added MPE discriminator calibration.  There are now two
  $<$discriminator$>$ elements, one with id=''spe'' (the previous SPE
  discriminator calibration), and the new MPE discriminator calibration
  with id=''mpe''.\\
  New: $<$discriminator id=''spe''$>$ and $<$discriminator id=''mpe''$>$
  record linear fits for SPE and MPE discriminator calibrations.\\
  Old: $<$discriminator$>$ calibration recorded only the the linear fit for
  the SPE discriminator calibration.\\

\item{6.0}\\ 
  Changed nominal value of front-end impedance based on measurements from
  LBNL.  Charge integration code must use this new value starting with this
  version of DOM-Cal calibration results.\\
  New: Nominal FE impedance is 43\ $\Omega$.\\
  Old: Nominal FE impedance was 50\ $\Omega$.\\

  Added FADC calibration.  Baseline (in FADC ticks) as a function of FADC
  front-end bias (DAC 10) is calibrated with a linear fit.  The gain of the
  FADC, including intrinsic gain and amplifiers is recorded along with
  the error of this mean gain value.  The units on this gain value are
  V/tick and should multiply the FADC value, unlike the ATWD channel gain
  calibrations.\\
  New: $<$fadc\_baseline$>$ linear fit records baseline as a function of
  DAC setting.  $<$fadc\_gain$>$ records gain value and error.\\
  Old: FADC placeholder elements, $<$fadc parname=''pedestal''
  value=''0''$>$, and $<$fadc parname=''gain'' value=''1''$>$.\\

  Added FADC time offset calibration.  Time offset in ns from ATWD waveform
  start is recorded, along with an error on this mean value.  Value is
  negative, reflecting the fact that FADC waveform starts earlier in time
  than the ATWD waveform.\\
  New: $<$fadc\_delta\_t$>$ records delta\_t value and error.\\
  Old: N/A\\

  Added discriminator calibration.  SPE discriminator trigger level
  (amplified charge, in pC) as a function of SPE discriminator DAC setting
  is recorded as a linear fit.\\
  New: $<$discriminator$>$ linear fit records calibration.\\
  Old: N/A\\

  Removed pulser calibration.  Pulser charge replaces discriminator level
  as primary uncalibrated input.\\
  New: N/A\\
  Old: $<$pulser$>$ linear fit recorded pulser amplitude as a function of
  pulser DAC setting.

\item{5.14}\\ 
  Changes to $<$atwdfreq$>$ calibration element.  Fit model is now
  quadratic instead of linear, and value resulting from the fit no longer
  needs to be multiplied by the clock frequency. \\
  New: {$<$fit model="quadratic"$>$}, with three fit coefficients and $R^2$.\\
  Frequency in MHz given by this relation:
  \be
  \mathrm{sampling\ speed\ (MHz)} = c_0 + c_1 \cdot \mathrm{DAC[0,4]} + c_2
  \cdot \mathrm{DAC[0,4]}^2\ .
  \ee
  Old: {$<$fit model="linear"$>$}, with two fit coefficients and $R^2$.\\
  Frequency in MHz given by this relation:
  \be
  \mathrm{sampling\ speed\ (MHz)} = 20.0\ (m \cdot \mathrm{DAC[0,4]} + b\ ).
  \ee

\item{5.13}\\ 
  Added attribute ``num\_pts'' to pmtTransitTime to indicate how
  many high voltage points were used for the linear fit (minimum of 2).\\
  New: $<$pmtTransitTime num\_pts=''9''$>$\\
  Old: $<$pmtTransitTime$>$

\ei

\end{document}
